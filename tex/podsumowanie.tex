\chapter{Podsumowanie}
\thispagestyle{chapterBeginStyle}
W pracy udało się spełnić wszystkie pierwotne wymagania. W jej wyniku został stworzony program, który pozwala na zamodelowanie na wejściu układu logicznego i tworzy odpowiadający mu układ kwantowy. Program symuluje również działanie układu kwantowego dla pewnego podzbioru bramek kwantowych.
\section{Minimalizacja liczby kubitów pomocniczych}
Głównym celem optymalizacyjnym w tej pracy było generowanie układów kwantowych w sposób, który minimalizuje używane kubity pomocnicze. Program w tym celu wykorzystuje algorytm oparty o uzykanie postaci ESOP dla funkcji boolowskich, a następnie zamianę tej postaci na ciąg bramek z rodziny Toffoliego oraz bramek $NOT$. W wyniku zastosowania tej metody wykorzystywany jest tylko jeden kubit pomocniczy dla każdej wyjściowej funkcji boolowskiej, przechowujący wartość funkcji na wyjściu.
\subsection{Wnioski i uwagi}
Podczas przeprowadzania obliczeń kwantowych bardzo istotnym czynnikiem jest czas działania. Zjawisko kwantowej dekoheracji powoduje, że czym dłużej obliczenia trwają, tym mniej dokładne zwracają wyniki.
\subsubsection{Optymalizacja postaci ESOP}
Zwracana przez kolejne rozwinięcia Shannona postać ESOP często nie jest najbardziej optymalną. Istnieją heurystyki, które służą optymalizacji postaci ESOP funkcji boolowskich (np. Exorcism-MV-2 \cite{Exorcism2}). Głównym celem tych optymalizacji jest minimalizacja liczby termów wykorzystywanych do zapisu funkcji.
\subsubsection{Rodzina bramek Toffoliego}
Z każdą bramką kwantową związany jest pewnien koszt. Koszt ten jest zależny między innym od wielkośći bramki. Układy kwantowe pozwalają z reguły na wykorzystywanie tylko niewielkiego zbióru bramek kwantowych, ograniczając się to tych, które operują na maksymalnie dwóch kubitach. W wyniku tego wielokubitowe bramki Toffoliego muszą zostać rozbite na ciąg mniejszych bramek. Możliwe jest rozbicie n-kubitowej bramki Toffoliego na maksymalnie dwukubitowe \cite{Saeedi_2013}, nie dodając bitów pomocniczych. Prowadzi to jednak do znacznego wzrostu iczby wykorzystywanych bramek.
\subsubsection{Kubity pomocnicze}
Największe istniejące komputery kwantowe nie przekraczają rozmiaru kilkudziesięciu kubitów, zatem łatwo widać, że kubity są cennym zasobem podczas obliczeń i warto minimalizować ich użycie. Żeby jednak komputery kwantowe mogły pracować i wykonywać realne obliczenia, liczba kubitów musi znacznie wzrosnąć. 
\par Stan pierwotny kubitów pomocniczych, które przechowują niestotne wyniki, może zostać przywrócony dzięki odwacalności obliczeń kwantowych. Zatem te same kubity mogą być używane wielokrotnie i można sobie wyobrazić, że jednostki obliczeniowe będą posiadały zasób w postaci pewnej liczby kubitów pomocniczych i wykorzystywały je w rzie ptrzeby podczas obliczeń.
\section{Dalszy rozwój}
\subsection{Poszerzenie zbioru bramek kwantowych}
Dobrym kierunkiem rozwojowym powstałego podczas tworzenia tej pracy programu byłoby poszerzenie zbioru bramek kwantowych, których użycie umożliwia. W szczególności dla logiki istotna mogłaby być bramka $\sqrt{NOT}$ i inne bramki pozwalające na rozbicie bramek Toffoliego na mniejsze.
\subsection{Dodanie nowych instrukcji}
Poza dodaniem większej liczby bramek kwantowych warto byłoby rozszerzyć język wejściowy. Umożliwić operatory infiksowe jak $\land$ w miejsce funkcji $and$. Można byłoby też stworzyć mechanizm definiowania funkcji na abstrakcyjnych zmiennych. W przyszłości może nawet deklarację wielu rejestrów, zarówno kwantowych jak i klasycznych. 