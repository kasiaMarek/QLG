\chapter{Analiza problemu}
\thispagestyle{chapterBeginStyle}
\label{rozdzial1}
\section{Założenia funkcjonalne}
Celem tej pracy jest stworzenie programu, który przyjmuje opis obliczeń (używając predefiniowanego języka) dla układu kwantowego. Natępnie tworzy układ kwantowy odpowidaj tym obliczeniom i symuluje jego działanie.
\subsection{Założenia dotyczące opisu wejściowego}
Opis wejściowy (język wejściowy) powinien pozwalać na niżej opisane działania.
\begin{enumerate}
    \item Zefiniowanie zmiennej oraz nadanie jej wartości początkowej (0 lub 1).
    \item Zdefiniowanie dowolnej funkcji boolowskiej na wcześniej zdefiniowanych zmiennych.
    \item Wprowadzenie zmiennej w stan superpozycji, taki, że prawdopodobieństwo otrzymania 0 w wyniku mierzenia wynosi $\frac{1}{2}$.
\end{enumerate}
\subsection{Założenia dotyczące zwracanego wyniku}
Program powinien zwrócić
\begin{enumerate}
    \item Układ kwantowy w postaci stanu początkowego rejestru oraz serii bramek.
    \item Wynik symulacji.
\end{enumerate}
\section{Założenia niefunkcjonalne}
Program powinien starać się optymalizować liczbę używanych dodatkowych kubitów, wymaganych przy przełożeniu funkcji nieodwracalnych na bramki kwantowe.
