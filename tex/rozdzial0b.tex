\chapter{Model układu kwantowego}
\thispagestyle{chapterBeginStyle}
\label{rozdzial0b}

\section{Stan kwantowy kubitu}
Podstawowym nośnikiem informacji w obliczeniach kwantowych, analogicznym do klasycznego bitu, jest bit kwantowy nazywany kubitem. Stan $\ket{0}$ odpowiada klasycznemu 0, a $\ket{1}$ klasycznej 1. Stany $\ket{0}$ i $\ket{1}$ to stany czyste. Kubit może znajdować się w superpozycji tych stanów, w stanie mieszanym. 
\par Mierząc stan kubitu można odkryć jedynie, że jest on $\ket{0}$ lub $\ket{1}$. Stan kubitu po jego zmierzeniu, jeśli był mieszany, ulegnie zmianie i będzie zgodny ze stanem zmierzonym.
\par Stan kubitu można zapisać jako
\[\q = \alpha_0\ket{0} + \alpha_1\ket{1}\]
gdzie $\alpha_0, \alpha_1 \in \CC$ oraz $\left|\alpha_0\right|^2 + \left|\alpha_1\right|^2 = 1$. Zapis ten mówi o prawdobodobieństwie otrzymania każdego z możliwych wyników przy pomiarze. Prawdopodobieństwo otrzymania wyniku $\ket{0}$ wynosi $\left|\alpha_0\right|^2$, a $\ket{1}$ wynosi $\left|\alpha_1\right|^2$.
\subsection{Notacja Diraca}
Zapis '$\ket{}$' nazywany jest notacją Diraca lub bra-ket. 
Przez $\ket{\psi}$ (nazywanym ket) rozumiemy pewien wektor kolumnowy w przestrzeni Hilberta, gdzie przestrzeń Hilberta jest przestrzenią wektorową nad ciałem liczb zespolonych ze zdefiniowanym iloczynem skalarnym.
Wektory 
\[
    \ket{0}
    =
    \begin{bmatrix}
        1 \\
        0 \\
    \end{bmatrix}
\], 
\[
    \ket{1}
    =
    \begin{bmatrix}
        0 \\
        1 \\
    \end{bmatrix}
\] tworzą bazę ortonormalną w 2-wymiarowej przestrzeni Hilberta.
\par Zatem stan kubitu odpowiada znormalizowanymu wektorowi w 2-wymiarowej przestrzeni Hilberta.
\section{Stan kwantowy rejestru kubitów}
Stan n-kubitowy rejestru kwantowego to
\[
    \ket{\psi} 
    = 
    \alpha_0 \ket{0} + \alpha_1 \ket{1} + \ldots + \alpha_{2^{n} - 1} \ket{2^{n} - 1}
    =
    \begin{bmatrix}
        \alpha_{0} \\
        \alpha_{1} \\
        \vdots \\
        \alpha_{n}
    \end{bmatrix}
\]
Gdzie ($\forall i \in \{0, 1, \ldots, 2^{n} - 1 \}$)($\alpha_i \in \CC$) oraz
\[\sum^{2^{n} - 1}_{i = 0} \left|\alpha_i\right|^{2} = 1\]
\subsection{Stan rejestu kubitów a stanów kubitów}
Mając rejestr kwantowy $\ket{\psi}$ , którego składowe stany kubitów to $\ket{q_0}$ i $\ket{q_1}$, gdzie
\[\ket{q_0} = \alpha_{00}\ket{0} + \alpha_{01}\ket{1}\]
\[\ket{q_1} = \alpha_{10}\ket{0} + \alpha_{11}\ket{1}\]
Możemy zapisać stan rejestru jako
\[\ket{\psi} = \ket{q_0} \otimes \ket{q_1} = \alpha_{00}\alpha_{10}\ket{00} + \alpha_{00}\alpha_{11}\ket{01} + \alpha_{01}\alpha_{10}\ket{10} + \alpha_{01}\alpha_{11}\ket{11}\]
gdzie przez $\otimes$ rozumiemy produkt tensorowy.
\par Analogicznie dla rejestru n-kubitowego $\ket{\psi}$, którego stany kubitów to $\ket{q_0}, \ket{q_1}, \ldots \ket{q_n}$
\[\ket{\psi} = \ket{q_0} \otimes \ket{q_1} \otimes \ldots \otimes \ket{q_n}\]
\subsection{Splątanie kwantowe}
N-kubitowy rejestr kwantowy może być w stanie $\ket{\psi}$ takim, że nie istnieją takie stany $\ket{q_0}, \ket{q_1}, \ldots \ket{q_n}$, że
\[\ket{\psi} = \ket{q_0} \otimes \ket{q_1} \otimes \ldots \otimes \ket{q_n}\]
Zjawisko to nazywane jest splątaniem kwantowym.
\par Przykładem takiego stanu jest stan Bella
\[\ket{\psi} = \frac{1}{\sqrt{2}}\ket{00} + \frac{1}{\sqrt{2}}\ket{11}\]
\section{Bramka kwantowa}
Przez bramkę kwantową rozumiemy operację (o dodatkowych ograniczeniach) wykonywaną na rejestrze kubitów, której wynikiem jest nowy stan rejestru kubitów. Ich działanie jest analogiczne do bramek w układach logicznych.
\par Każda poprawna bramka kwantowa może zostać zapisana jako macierz unitarna $M$, to znaczy taka, że $M^{\dagger}M = I$ oraz $MM^{\dagger} = I$, gdzie przez $\dagger$ rozumiemy sprzężenie hermitowskie (złożenie operacji transpozycji i sprzężenia zespolonego). 
N-kubitowa bramka kwantowa $M_n$ jest zatem macierzą unitarną o wymiarach $2^n \times 2^n$ i co więcej każda taka macierz jest poprawną bramką kwantową.
\par N-kubitowa bramka kwantowa $M$ operuje na n-kubitowym rejestrze kwantowym $\q$ w następujący sposób:
\begin{align*}
    M\q = M\ket{a_0, a_1, \ldots a_{2^n-1}}
    &= M(a_0\ket{0} + a_0\ket{1} + \ldots + a_{2^n-1}\ket{2^n-1}) \\
    &= a_0*M\ket{0} + a_0*M\ket{1} + \ldots + a_{2^n-1}*M\ket{2^n-1}
\end{align*}
\section{Wybrane bramki kwantowe}
\subsection{Bramka NOT}
Bramka NOT dokonuje mapowania
\[\ket{0} \rightarrow \ket{1}\]
\[\ket{1} \rightarrow \ket{0}\]
analogicznie do logicznej bramki Not.
\par Macierzą odpowiadającą temu mapowaniu jest
\[
    X
    =
    \begin{bmatrix}
        0 & 1 \\
        1 & 0 \\
    \end{bmatrix}
\]
Macierz ta jest unitarna.\\
Bramka Not operuje na kubicie $\q$ w natępujący sposób
\[
    X\q 
    =
    X(\Q)
    =
    \begin{bmatrix}
        0 & 1 \\
        1 & 0 \\
    \end{bmatrix}
    \begin{bmatrix}
        \alpha_{0} \\
        \alpha_{1} \\
    \end{bmatrix}
    =
    \begin{bmatrix}
        \alpha_{1} \\
        \alpha_{0} \\
    \end{bmatrix}
\]
\par Bramkę NOT zapisuje się na układzie kwantowym w następujący sposób
\[
\Qcircuit @C=1.5em @R=1.5em {
    \lstick{\ket{a}} & \targ & \rstick{\ket{\neg a}} \qw
}
\]
\subsection{Uogólniona bramka Toffoliego}
\subsubsection{Bramka Feymana}
Bramka Faymana inaczej $CNOT$ (z ang. Controlled NOT) jest bramką operującą na 2 kubitach. Dokonuje następującego mapowania 
\[\ket{00} \rightarrow \ket{00}\]
\[\ket{01} \rightarrow \ket{01}\]
\[\ket{10} \rightarrow \ket{11}\]
\[\ket{11} \rightarrow \ket{10}\]
Inaczej jej działanie można zapisać w postaci
\[
    CNOT\ket{a,b} = \ket{a,b \oplus a}
\]
Wtedy można patrzeć na nią jak na uogólnioną bramkę $XOR$.\\
Bramce tej odpowiadają następująca macierz i zapis w układzie kwantowym:
\begin{paracol}{2}
\[
    CNOT
    =
    \begin{bmatrix}
        1 & 0 & 0 & 0 \\
        0 & 1 & 0 & 0 \\
        0 & 0 & 0 & 1 \\
        0 & 0 & 1 & 0 \\
    \end{bmatrix}
\]
\switchcolumn
\vspace*{\fill}
\[
    \Qcircuit @C=1.5em @R=1.5em {
        \lstick{\ket{a}} & \ctrl{1} & \rstick{\ket{a}} \qw \\
        \lstick{\ket{b}} & \targ & \rstick{\ket{b \otimes a}} \qw
    }
\] 
\vspace*{\fill}
\end{paracol}
\subsubsection{Bramka Toffoliego}
Bramką analogiczną do bramki Feymana jest bramka Toffoliego, inaczej $CCNOT$. Nakłada NOT na ostatni kubit (bit wejściowy), jeśli dwa pozostałe (bity sterujące) są w stanie $\ket{11}$. Jej działanie można zapisać następująco:
\[
    CCNOT\ket{a,b,c} = \ket{a,b,c \oplus ab}
\]
Macierz oraz zapis na układzie kwantowym dla tej bramki to:
\begin{paracol}{2}
    \[
        CCNOT
        =
        \begin{bmatrix}
            1 & 0 & 0 & 0 & 0 & 0 & 0 & 0 \\
            0 & 1 & 0 & 0 & 0 & 0 & 0 & 0 \\
            0 & 0 & 1 & 0 & 0 & 0 & 0 & 0 \\
            0 & 0 & 0 & 1 & 0 & 0 & 0 & 0 \\
            0 & 0 & 0 & 0 & 1 & 0 & 0 & 0 \\
            0 & 0 & 0 & 0 & 0 & 1 & 0 & 0 \\
            0 & 0 & 0 & 0 & 0 & 0 & 0 & 1 \\
            0 & 0 & 0 & 0 & 0 & 0 & 1 & 0 \\
        \end{bmatrix}
    \]
    \switchcolumn
    \vspace*{\fill}
    \[
        \Qcircuit @C=1.5em @R=1.5em {
            \lstick{\ket{a}} & \ctrl{1} & \rstick{\ket{a}} \qw \\
            \lstick{\ket{b}} & \ctrl{1} & \rstick{\ket{b}} \qw \\
            \lstick{\ket{c}} & \targ & \rstick{\ket{c \otimes ab}} \qw
        }
    \]
    \vspace*{\fill}
\end{paracol}
\subsubsection{Uogólniona forma}
Bramkę Toffoliego można uogólnić do bramki z n-bitami kontolującymi $C_nNOT$.
\[
    C_nNOT\ket{a_0, a_1, \ldots, a_n} = C_nNOT\ket{a_0, a_1, \ldots, a_n \otimes a_0a_1\ldots a_{n-1}}
\]
Wtedy dla $n = 2$ mamy bramkę Toffoliego, dla $n = 1$ bramkę Feymana, a dla $n = 0$ bramkę $NOT$.
\subsection{Bramka SWAP}
Bramka $SWAP$ operuje na dwóch kubitach zamieniając ich stany.
\[
    SWAP\ket{a,b} = \ket{b,a}
\]
Macierz oraz zapis na układzie kwantowym dla tej bramki to:
\begin{paracol}{2}
    \[
        SWAP = 
        \begin{bmatrix}
            1 & 0 & 0 & 0 \\
            0 & 0 & 1 & 0 \\
            0 & 1 & 0 & 0 \\
            0 & 0 & 0 & 1 \\
        \end{bmatrix}
    \]
    \switchcolumn
    \vspace*{\fill}
    \[
        \Qcircuit @C=1.5em @R=1.5em {
        \lstick{\ket{a}} & \qswap & \rstick{\ket{b}} \qw \\
        \lstick{\ket{b}} & \qswap & \rstick{\ket{a}} \qw
        }
    \]
    \vspace*{\fill}
\end{paracol}
\subsection{Bramka Fredkina}
3-kubitowa bramka Fredkina inaczej $CSWAP$ dokonuje permutacji ostatnich dwóch bitów, wtedy i tylko wtedy gdy bit sterujący jest w stanie $\ket{1}$. Jej działanie można zapisać w następujący sposób:
\[
    CSWAP\ket{c,s_1,s_2} = \ket{c, (\neg c) s_1 + c s_2, (\neg c)s_2 + c s_1}
\]
Macierz oraz zapis układzie kwantowym dla tej bramki to:
\begin{paracol}{2}
    \[
        CSWAP = 
        \begin{bmatrix}
            1 & 0 & 0 & 0 & 0 & 0 & 0 & 0\\
            0 & 1 & 0 & 0 & 0 & 0 & 0 & 0\\
            0 & 0 & 1 & 0 & 0 & 0 & 0 & 0\\
            0 & 0 & 0 & 1 & 0 & 0 & 0 & 0\\
            0 & 0 & 0 & 0 & 1 & 0 & 0 & 0\\
            0 & 0 & 0 & 0 & 0 & 0 & 1 & 0\\
            0 & 0 & 0 & 0 & 0 & 1 & 0 & 0\\
            0 & 0 & 0 & 0 & 0 & 0 & 0 & 1\\
        \end{bmatrix}
    \]
    \switchcolumn
    \vspace*{\fill}
    \[
        \Qcircuit @C=1.5em @R=1.5em {
        \lstick{\ket{a}} & \ctrl{2} & \rstick{\ket{a}} \qw \\
        \lstick{\ket{b}} & \qswap & \rstick{\ket{b'}} \qw \\
        \lstick{\ket{c}} & \qswap & \rstick{\ket{c'}} \qw
        }
    \]
    \vspace*{\fill}
\end{paracol}
\subsection{Bramka Hadamarda}
Wszystkie wyżej opisane bramki można zaimplementować na klasycznych bitach. Najbardziej podstawowa bramka, która wpowadza kubit w stan superpozycji, to bramka Hadamarda. Dokonuje ona następującego mapowania:
\[H\ket{0} = \frac{1}{\sqrt{2}}\ket{0} + \frac{1}{\sqrt{2}}\ket{1} = \ket{+}\]
\[H\ket{1} = \frac{1}{\sqrt{2}}\ket{0} - \frac{1}{\sqrt{2}}\ket{1} = \ket{-}\]
Zarówno w stanie $\ket{+}$ jak i w stanie $\ket{-}$
\[P(\ket{0}) = P(\ket{1}) = \frac{1}{2}\]
gdzie przez $P(\q)$ rozumiemy prawdopodobieństwo otrzymania $\q$ w wyniku pomiaru.
\par Bramkę Hadamarda można zapisać jako następującą macierz
\begin{paracol}{2}
    \[
        H
        = \frac{1}{\sqrt{2}}
        \begin{bmatrix}
            1 & 1 \\
            1 & -1 \\
        \end{bmatrix}
    \]
    \switchcolumn
    \vspace*{\fill}
    \[
        \Qcircuit @C=1.5em @R=1.5em {
            \lstick{\ket{a}} & \gate{H} & \rstick{H\ket{a}} \qw
        }
    \]
    \vspace*{\fill}
\end{paracol}
\section{Układ kwantowy}
\begin{definition}
    Przez układ kwantowy rozumiemy ciąg bramek kwantowych, które są zdefiniowane jako operacje na $n$-bitowym rejetrze z nałożonym porządkiem ich wykonywania. Każda z bramek kwantowych wykonywana jest na podzbiorze kubitów z rejestru. Każdy układ kwantowy można wyrazić jako pojedynczą operację, która będącą złożeniem operacji składowych. Operacja ta, jako złożenie macierzy unitarnych, również będzie macierzą unitarną.
\end{definition}