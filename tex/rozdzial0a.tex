\chapter{Logika klasyczna}
\thispagestyle{chapterBeginStyle}
\label{rozdzial0a}
\section{Funkcja boolowska}
\begin{definition}
    Przez funkcję boolowską, w tej pracy, rozumiemy zupełną funkcję boolowską. Jest to funkcja
    \[f:\{0,1\}^n \rightarrow \{0,1\}\]
    gdzie $n \in \mathbb{N}$. Wartość 1 jest utożsamiana z logiczną prawdą, a 0 z logicznym fałszem.
\end{definition}
\par Funkcje boolowska możemy zapisać w postaci tabeli prawdy. Na przykład
\begin{center}
    \begin{tabular}{| c  c  c | c |}
        \hline
        x & y & z & $f(x,y,z)$ \\ 
        \hline
        0 & 0 & 0 & 0 \\ 
        0 & 0 & 1 & 1 \\ 
        0 & 1 & 0 & 1 \\ 
        0 & 1 & 1 & 0 \\ 
        1 & 0 & 0 & 0 \\ 
        1 & 0 & 1 & 1 \\ 
        1 & 1 & 0 & 0 \\ 
        1 & 1 & 1 & 0 \\ 
        \hline 
    \end{tabular}
\end{center}
wtedy $f(0, 1, 0) = 1$, a $f(0,0,0) = 0$.
\subsection{Podstawowe funkcje boolowskie}
Poniższy zestaw funkcji boolowskich tworzy układ funkcjonalnie pełny.
\begin{definition}
    Zestaw funkcji boolowskich tworzy układ funkcjonalnie pełny, gdy przy ich użyciu można wyrazić każdą funkcję boolowską.
\end{definition}
\begin{multicols}{2}
\subsubsection{Id (identyczność)}
\[f(a) = a\]
\begin{center}
    \begin{tabular}{| c | c |}
        \hline
        x & $f(x) = x$ \\ 
        \hline
        0 & 0 \\ 
        1 & 1 \\ 
        \hline 
    \end{tabular}
\end{center}
\subsubsection{Not (negacja)}
\[f(a) = \neg a\]
\[f(a) = \overline{a}\]
\begin{center}
    \begin{tabular}{| c | c |}
        \hline
        x & $f(x) = \overline{x}$ \\ 
        \hline
        0 & 1 \\ 
        1 & 0 \\ 
        \hline 
    \end{tabular}
\end{center}
\end{multicols}
\begin{multicols}{2}
\subsubsection{And (koniunkcja)}
\[f(a,b) = a \land b\]
\[f(a,b) = ab\]
\begin{center}
    \begin{tabular}{| c  c | c |}
        \hline
        x & y & $f(x,y) = xy$ \\ 
        \hline
        0 & 0 & 0 \\ 
        0 & 1 & 0 \\ 
        1 & 0 & 0 \\ 
        1 & 1 & 1 \\ 
        \hline 
    \end{tabular}
\end{center}
\subsubsection{Or (alternatywa)}
\[f(a,b) = a \lor b\]
\[f(a,b) = a + b\]
\begin{center}
    \begin{tabular}{| c  c | c |}
        \hline
        x & y & $f(x,y) = x + y$ \\ 
        \hline
        0 & 0 & 0 \\ 
        0 & 1 & 1 \\ 
        1 & 0 & 1 \\ 
        1 & 1 & 1 \\ 
        \hline 
    \end{tabular}
\end{center}
\end{multicols}
\begin{paracol}{2}
\subsubsection{Xor (alternatywa wykluczająca)}
\[f(a,b) = a \otimes b\]
\begin{center}
    \begin{tabular}{| c  c | c |}
        \hline
        x & y & $f(x,y) = x \otimes y$ \\ 
        \hline
        0 & 0 & 0 \\ 
        0 & 1 & 1 \\ 
        1 & 0 & 1 \\ 
        1 & 1 & 0 \\ 
        \hline 
    \end{tabular}
\end{center}
\switchcolumn
\vfill{}
\end{paracol}
\subsection{Kanoniczna postać sumacyjna (SOP)}
\begin{definition}
Literał definiuje się jako:
$$
x^e = \left\{ \begin{array}{ll}
x & \textrm{gdy $ e = 1$}\\
\overline{x} & \textrm{gdy $ e = 0 $}
\end{array} \right.
$$
gdzie $e \in \{0, 1\}$, a $x$ symbolem zmiennej.
\end{definition}
\begin{definition}
    Term to produkt literałów (równoważny logicznej koniunkcji), taki, że każda zmienna występuje w nim maksymalnie raz.
\end{definition}
\begin{definition}
    Term, w którym każda zmienna występuje dokładnie raz, nazywamy mintermem.
\end{definition}
\begin{definition}
    Kanoniczną postacią sumacyjną (SOP) nazywamy postać funkcji, w której zapisana jest ona jako suma termów (równoważna logicznej alternatywie).
\end{definition}
Każdą funkcję boolowską można zapisać w postaci sumy mintermów w następujący sposób
\[f(a_0, a_1, \ldots a_n) = \sum_{i=0}^n b_i*m_i\]
gdzie $b_i \in \{0, 1\}$ jest wskaźnikiem (mówi czy dany minterm należy do funkcji f), a $m_i$ to minterm, taki że
\[m_i = a_0^{j_0}a_1^{j_1} \ldots a_n^{j_n}\]
gdzie ciąg $j_0, j_1, \ldots j_n$ to cyfry liczby $i$ zapisanej w postaci binarnej.
\section{Układy logiczne}
Układ logiczny $U_g$ pozwala na obliczanie funkcji $g: \{0,1\}^n \rightarrow \{0,1\}^k$, gdzie $n,k \in \mathbb{N}$. Zatem układ ten można zamodelować następująco:
\begin{center}
    \begin{tabular}{ c | c | c }
        \cline{2-2}
        & & \\
        $a_0 \rightarrow$ & \multirow{4}{5cm}{\centering$g$ 
        } & $\rightarrow o_0 $
        \\
        $a_1 \rightarrow$ &  & $\rightarrow o_1 $
        \\
        \vdots & & \vdots\\
        $a_n \rightarrow$ & & $\rightarrow o_k $
        \\
        & & \\
        \cline{2-2}
    \end{tabular} 
\end{center}
gdzie ciągi $a_0, a_1, \ldots, a_n$ to bity wejściowe, a $b_0, b_1 \ldots, b_n$ to bity wyjściowe, że $\forall i$ $a_i, b_i \in \{0, 1\}$ oraz
\[g(a_0, a_1, \ldots, a_n) = (f_0(a_0, a_1, \ldots, a_n), f_1(a_0, a_1, \ldots, a_n), \ldots f_k(a_0, a_1, \ldots, a_n)) =  (o_0, o_1, \ldots, o_k)\]
gdzie $f_0, f_1, \ldots, f_k$ to funkcje boolowskie.
