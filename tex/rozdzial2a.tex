\chapter{Translacja układów logicznych do układów kwantowych}
\thispagestyle{chapterBeginStyle}
\label{rozdzial2a}
Jednym z najważniejszych zagadnień, którymi zajmuje się ta praca, jest translacja układów logicznych, rozumianych jako zestaw funkcji boolowskich (rodział \ref{rozdzial0a}), do układów kwantowych. Celem przedstawionego tu problemu translacji jest stworzenie układu kwantowego, który oblicza te same funkcje boolowskie co zadany układ, z wykorzystaniem zestawu bramek przedstawionego wwe wcześniejszych rozdziałach, czyli bramek należących do rodziny bramek Toffoliego, bramki $SWAP$ oraz bramki Fredkina. 
\section{Problem translacji}
W przeciwieństwie do bramek logicznych, operacje składające się na układ kwantowy muszą być unitarne. W szczególności znaczy to, że przyjmują na wejściu tyle samo bitów ile zwracają na wyjściu. Zatem chcąc wyrazić te same operacje, które wykonuje układ logiczny za pomocą układu kwantowego, traktowanego tutaj jako pojedyncza operacja, należałoby odpowiednio zwiększyć liczbę bitów na wejściu/wyjściu i następnie zignorować te "dodatkowe" bity na wejściu/wyjściu. Nie zapewnia to jednak, że tak stworzone mapowanie przedstawione jako macierz będzie poprawną bramkę kwantową (tzn. czy macierz ta będzie unitarna).
\par Weźmy na przykład bramkę układ logiczny składający się z pojedynczej bramki $AND$. Żeby zachować stałą liczbę bitów na wejściu i wyjściu możemy stworzyć funkcję:
\[f_{AND}(a,b) = (0, ab)\]
Tak stworzona fukcja nie jest nawet odwracalna, co jest warunkiem koniecznym unitarności. Co więcej nie jestnieje taka funckcja boolowska $h$
\[f_{AND}(a,b) = (h(a,b), ab)\]
żeby $f_{AND}$ było funkcją odwracalną.
\par Dlatego symulowanie układów logicznych za pomocą układów kwantowych wymaga czasami wprowadzenia bitów pomocniczych, które stanowią dodatkową przestrzeń do obliczeń i których wartość jest początkowa jest znana, tutaj zawsze $\ket{0}$.
\section{Podejście naiwne}
\label{naive}
Jeżeli jednak stworzony układ kwantowy nie jest pojedynczą bramką, a ciągiem operacji, to każda z tych operacji musi być uniratna. Oznacza to w szczególności, że nie tylko liczbą bitów na wejściu do układu i wyjściu z układu muszą być sobie równe, ale że liczba bitów, na których bramki kwantowe operują jest stała. W szczególności nie można dokonać "rodzielenia kabla" jak to się dzieje w przypadku układów klasycznych.
\par Przykładowym rozwiązaniem tego problemu mogłoby być wprowadzenie rozszerzenie rejestru kubitów, tak by każdej zmiennej odpowiadała Dlatego o każdej takiej opracji będziemy myśleć jako o bramce $COPY$. Której zachowanie można zapisać za pomocą funkcji następująco:
\[COPY(a) = (a,a)\] 
\par Jeżeli potrafimy wyrazić bramkę $COPY$ oraz zestaw bramek logicznych tworzących układ funkcjonalnie zupełny za pomocą bramek kwantowych to potrafimy wyrazić każdy układ logiczny za pomocą układu kwantowego.
\par Zatem naturalnie pierwszym podejściem do problemu translacji układów logicznych na kwantowe jest wyrażenie tego układu używając jedynie pewnego ograniczonego (zupełnego) zestawu bramek, na przykład $COPY, NOT, AND$, a następnie bezpośrednia zamiana każdej z tych operacji na bramki kwantowe.
\subsection{Copy}
Ze względu na stałą liczbę kubitów w układzie kwantowym, każda bramka $COPY$ wymaga powiększenia rejestru wejściowego o jeden bit. Zatem "kwantowa bramka $COPY$" będzie dokonywała mapowania 
\[C(a,0) = (a,a)\]
dla $a \in \{0,1\}$.
\par Operację te można wyrazić za pomocą bramki Feymana
\[
    \Qcircuit @C=1.5em @R=1.5em {
        \lstick{\ket{a}} & \ctrl{1} & \rstick{\ket{a}} \qw \\
        \lstick{\ket{0}} & \targ & \rstick{\ket{0 \otimes a} = \ket{a}} \qw
    }
\]
\subsection{Not}
Logicznej opracji $NOT$ odpowiada bramka kwantowa $NOT$.
\subsection{And}
Weźmy funkcję
\[T(a,b,c) = (a,b, c \otimes ab)\]
Wtedy jeżeli $c = 0$ to
\[T(a,b,0) = (a,b, ab)\]
Operacji tej odpowiada bramka Toffoliego:
\[
    \Qcircuit @C=1.5em @R=1.5em {
        \lstick{\ket{a}} & \ctrl{1} & \rstick{\ket{a}} \qw \\
        \lstick{\ket{b}} & \ctrl{1} & \rstick{\ket{b}} \qw \\
        \lstick{\ket{0}} & \targ & \rstick{\ket{0 \otimes ab} = \ket{ab}} \qw
    }
\]
Zatem po wprowadzeniu bitu pomocniczego, można obliczyć $AND$ za pomocą bramki Toffoliego.
\section{Wyrażanie funkcji boolowskich za pomocą układów kwantowych}
\subsection{Wejście}
W tej pracy, zgodnie z modelem przedstawionym w rozdziale \ref{rozdzial0a}, myślimy jako o wektorze funkcji boolowskich. Zatem wejściem do translatora jest funkcja boolowska, ponieważ będzimy zajmować się "jedną na raz". Dodatkowo zakładamy, że każda zadeklarowana zmienna jest również częścią wyjścia, zatem nie możemy "zatracić" wartości tych zmiennych.
\subsection{Przykład}
Weźmy przykładową funkcję
\[f(a,b,c) = b + a\overline{c} = b \lor (a \land (\neg c))\]
Korzystając z metod translacji opisanych w rozdziale \ref{naive}, można przestawić tę funkcję w następujący sposób:
\[
    \Qcircuit @C=1.5em @R=1.5em {
        \lstick{\ket{a}} & \qw & \ctrl{2} & \qw & \qw & \qw & \qw & \rstick{\ket{a}} \qw \\
        \lstick{\ket{b}} & \qw & \qw & \qw & \targ & \ctrl{2} & \targ & \rstick{\ket{b}} \qw \\
        \lstick{\ket{c}} & \targ & \ctrl{1} & \targ & \qw & \qw & \qw & \rstick{\ket{c}} \qw \\
        \lstick{\ket{0}} & \qw & \targ & \qw & \targ & \ctrl{1} & \targ & \rstick{\ket{a\overline{c}}} \qw \\
        \lstick{\ket{0}} & \qw & \qw & \qw & \qw & \targ & \targ & \rstick{\ket{b + a\overline{c}}} \qw
    }
\]
Na wejściu zostały dodane dwa bity. Używając strategii tłumaczenia każdej operacji $AND, OR, NOT$ na zestaw bramek kwantowych liczba potrzebnych kubitów pomocniczych u rośnie z sumaryczną liczbą bramek $OR$ i $AND$. Jeden bit na wejściu jest konieczny to przechowania wyniku funkcji, ale wartości przechowywane na reszcie z kubitów pomocniczych są niestotne.
\subsection{Teoretyczne minimum dla bitów pomocniczych}
\begin{lemma}
    \label{l:min}
    Z każdej funkcji boolowskiej $f(\vec{x}) = f(x_0, x_1, \ldots x_n)$, dla pewnego $n \in \mathbb{N}$ oraz $\vec{x} \in \{0, 1\}^n$, można stworzyć funkcję $f_Q : \{0, 1\}^{n+1}\rightarrow \{0, 1\}^{n+1}$, dla której 
    \[f_Q(x_0, x_1, \ldots x_n, 0) = (x_0, x_1, \ldots x_n, f(x_0, x_1, \ldots x_n))\] 
    oraz, której odpowiada macierz unitarna wymiarów $2^{n+1} \times 2^{n+1}$.
\end{lemma}
\begin{proof}
    Zdefiniujmy $f_Q$ następująco
    \[f_Q(x_0, x_1, \ldots x_n, c) = (x_0, x_1, \ldots x_n, c \otimes f(\vec{x}))\]
    Wtedy dla c = 0
    \[f_Q(x_0, x_1, \ldots x_n, 0) = (x_0, x_1, \ldots x_n, 0 \otimes f(\vec{x}))\]
    Funkcją przeciwną do $f_Q$ jest ona sama.
    \[f_Q(f_Q(x_0, x_1, \ldots x_n, c)) = (x_0, x_1, \ldots x_n, c \otimes f(\vec{x}) \otimes f(\vec{x})) = (x_0, x_1, \ldots x_n, c)\]
    Ponieważ funkcja $f_Q$ jest bijakcją można wyrazić ją jako permutację na wektorach $\{0,1\}^{n+1}$. Dla macierzy $M$ wyrażającej tę permutację otrzymujemy $\forall x_0, x_1, \ldots x_n \in \{0,1\}^{n+1}$
    \[MM\ket{x_0, x_1, \ldots x_n} = \ket{x_0, x_1, \ldots x_n}\]
    Stąd
    \[MM = I\]
\end{proof}
Z lematu \ref{l:min} wynika, że każdą funkcję boolowską można można wyrazić za pomocą układu kwantowego z użyciem tylko jednego bitu pomocniczego, który na wyjściu przechowuje wynik tej funkcji.
\section{Postać ESOP}
W rozdziale \ref{rozdzial0a} omówiona została sumacyjna postać kanoniczna fukcji boolowskiej. Funkcja w tej postaci jest wyrażona jako suma produktów (termów). Analogiczną postacią jest postać ESOP czyli alternatywa wykluczająca termów.
\begin{theorem}
    Każdą funkcję boolowską można zapisać w postaci ESOP.
\end{theorem}
\begin{proof}
    Weżmy funkcję boolowską $f: \{0, 1\}^n \rightarrow \{0, 1\}$ zapisaną w postaci sumy mintermów.
    \[f(a_0, a_1, \ldots a_n) = m_{k_0} + m_{k_1} + \ldots + m_{k_l}\]
    gdzie $m_i$ oznacza minterm, a $k_0, k_1, k_l \in \{0, 1, \ldots n\}$ oznaczają indeksy mintermów należących do funkcji f.\\
    Zauważmy następującą tożsamość
    \[a + b = a \otimes b \otimes ab\]
    Wtedy, ponieważ $\forall i,j$
    \[m_i \otimes m_j = 0\]
    funkcję $f$ można zapisać
    \[f(a_0, a_1, \ldots a_n) = m_{k_0} \otimes m_{k_1} \otimes \ldots \otimes m_{k_l}\]
\end{proof}
\section{Tworzenie układu kwantowego z postaci ESOP}
\begin{theorem}
    Każdą funkcję boolowską $f: \{0, 1\}^n \rightarrow \{0,1\}$ można obliczyć za pomocą układu, o rejestrze $n+1$-kubitowym,składającego się z maksymalnie $2^n$ uogólnionych bramek Toffoliego oraz $2^{n+1}*n$ bramek $NOT$.
\end{theorem}
\begin{proof}
    Weźmy funkcję $f$ w postaci alternatywy wykluczającej mintermów.
    \[f(a_0, a_1, \ldots a_n) = m_{k_0} \otimes m_{k_1} \otimes \ldots \otimes m_{k_l}\]
    Wtedy każdy minterm można obliczyć za pomocą maksymalnie $n*2$ bramek $NOT$ oraz uogólnionej bramki Toffoliego rozmiaru $n+1$. Pierwsze maksymalnie $n$ bramek $NOT$ jest używanych do otrzymania wartości literałów tworzących mineterm. Następnie bramka Toffoliego jest wykorzytana do obliczenia wartościowania mintermu i zapisania go na bicie pomocniczym. Pozostałe bramki $NOT$ są wykorzystane do przywrócenia stanów wejściowych na bitach argumentów.
    \par Dla przykładowego mintermu $m = x_0x_1\overline{x_2}\overline{x_3}x_4$ konstrukacja ta wygląda następująco:
    \columnratio{0.45, 0.1, 0.45}
    \begin{paracol}{3}
        \vspace*{\fill}
    \[
        \Qcircuit @C=1.5em @R=1.5em {
            \lstick{\ket{x_0}} & \qw & \ctrl{1} & \qw & \rstick{\ket{x_0}} \qw \\
            \lstick{\ket{x_1}} & \qw & \ctrl{1} & \qw & \rstick{\ket{x_1}} \qw \\
            \lstick{\ket{x_2}} & \qw & \ctrl{1} & \qw & \rstick{\ket{x_2}} \qw \\
            \lstick{\ket{x_3}} & \qw & \ctrl{1} & \qw & \rstick{\ket{x_3}} \qw \\
            \lstick{\ket{x_4}} & \qw & \ctrl{1} & \qw & \rstick{\ket{x_4}} \qw \\
            \lstick{\ket{0}} & \qw & \gate{m_i} & \qw & \rstick{\ket{m}} \qw
        }
    \]
    \vspace*{\fill}
    \switchcolumn
    \vspace*{\fill}
    \begin{center}
        =
    \end{center}
    \vspace*{\fill}
    \switchcolumn
    \vspace*{\fill}
    \[
        \Qcircuit @C=1.5em @R=1.5em {
            \lstick{\ket{x_0}} & \qw & \ctrl{1} & \qw & \rstick{\ket{x_0}} \qw \\
            \lstick{\ket{x_1}} & \qw & \ctrl{1} & \qw & \rstick{\ket{x_1}} \qw \\
            \lstick{\ket{x_2}} & \targ & \ctrl{1} & \targ & \rstick{\ket{x_2}} \qw \\
            \lstick{\ket{x_3}} & \targ & \ctrl{1} & \targ & \rstick{\ket{x_3}} \qw \\
            \lstick{\ket{x_4}} & \qw & \ctrl{1} & \qw & \rstick{\ket{x_4}} \qw \\
            \lstick{\ket{0}} & \qw & \targ & \qw & \rstick{\ket{m}} \qw
        }
    \]
    \vspace*{\fill}
    \end{paracol}
    \par Korzystając z tak zbudowanych framgentów układu obliczających wartości mintermów można stworzyć następujący układ kwantowy.
        \[
            \Qcircuit @C=1.5em @R=1.5em {
                & \lstick{\ket{x_0}} & \ctrl{1} & \ctrl{1} & \qw & \ldots & & \ctrl{1} & \rstick{\ket{x_0}} \qw \\
                & \lstick{\ket{x_1}} & \ctrl{1} & \ctrl{1} & \qw & \ldots & & \ctrl{1} & \rstick{\ket{x_1}} \qw \\
                \vdots & & \vdots & \vdots & & \ddots & & \vdots & & \vdots & \\
                & \lstick{\ket{x_n}} & \ctrl{1} & \ctrl{1} & \qw & \ldots & & \ctrl{1} & \rstick{\ket{x_n}} \qw \\
                & \lstick{\ket{0}} & \gate{m_{k_0}} & \gate{m_{k_1}} & \qw & \ldots & & \gate{m_{k_l}} & \rstick{\ket{((\ldots((0 \otimes m_{k_0}) \otimes m_{k_1}) \ldots )\otimes m_{k_l})}} \qw
            }
        \]
    Ostatni bit powyższego układu "zbiera" wartości mintermów. Jeśli na wejsciu był on zerem to na wyjściu będzie przechowywał wartość funkcji $f$.
\end{proof}
\subsection{Przykład}
Weźmy przykładową funkcję
\[f(a,b,c,d) = ab \otimes \overline{a}bc\overline{d} \otimes \overline{a}\overline{c}\]
w postaci alternatywy wykluczającej termów.
\par Wtedy układ kwantowy, zbudowany z bramek $NOT$ oraz bramek Toffoliego, wyrażający tę funkcję, wygląda następująco:
\[
    \Qcircuit @C=1.5em @R=1.5em {
        \lstick{\ket{a}} & \ctrl{1} & \targ & \ctrl{1} & \targ & \targ &  \ctrl{2} & \targ & \rstick{\ket{a}} \qw \\
        \lstick{\ket{b}} & \ctrl{3} & \targ & \ctrl{1} & \targ & \qw &    \qw &      \qw & \rstick{\ket{b}} \qw \\
        \lstick{\ket{c}} & \qw &      \qw &   \ctrl{1} & \qw &   \targ &  \ctrl{2} & \targ & \rstick{\ket{c}} \qw \\
        \lstick{\ket{d}} & \qw &      \qw &   \ctrl{1} & \qw &   \qw &    \qw &       \qw & \rstick{\ket{d}} \qw \\
        \lstick{\ket{0}} & \targ &    \qw &   \targ &    \qw &   \qw &    \targ &     \qw & \rstick{\ket{f}} \qw 
    }
\]
\subsection{Algorytm}
W celu optymalizacji bramek $NOT$ wykorzystanych w układzie obliczającym funkcję $f$ algorytm przetrzymuje informację o aktualnym stanie zmiennych. To znaczy informację czy kubit odpowiadający danej zmiennej przechowuje jej wartość czy negację tej wartości. Dzięki temu wartości argumentów nie są przywracane po każdej bramce Toffoliego, ale jedynie raz na sam koniec.\\
\begin{pseudokod}[H]
    \SetArgSty{normalfont}
    \KwIn{Zbiór termów funkcji w postaci ESOP $E$, index wyjścia $o$}
    \KwOut{Lista bramek kwantowych $G$}
    $negPol \leftarrow$ pusta lista \tcp*{zbiór aktualnie zagenowanych zmiennych}
    $G \leftarrow$ pusta lista \;
    \ForEach{$t \in E$}{
        \ForEach{$v \in t$} { 
            \If {$v$ postaci $\overline{x}$, gdzie x to zmienna}{
                \If { $v.id \notin negPol$} {
                    do $G$ dodaj \texttt{not}($v.id$) \;
                    do $negPol$ dodaj $v.id$ \;
                }
            } \Else {
                \If { $v.id \in negPol$} {
                    do $G$ dodaj \texttt{not}($v.id$) \;
                    z $negPol$ usuń $v.id$ \;
                }
            }
        }
        do $G$ dodaj \texttt{tfl}(lista $id$ elementów z $t$)($o$) \;
    }
    \tcc{przywrócenie stanu początkowego}
    \ForEach{$i \in negPol$}{ 
        do $G$ dodaj \texttt{not}($i.id$) \;
    }
    \caption{Konwersja postaci ESOP to listy bramek kwantowych}\label{alg:esop}
\end{pseudokod}
\section{Rozwinięcie Shannona}
Wejściem do tranlatora są funkcje boolowskie zdefiniowane za pomocą bramek logicznych $AND$, $NOT$, $OR$ oraz $XOR$. Zatem, że przekształcić każdą z tych funkcji na serię bramek kwantowych muszą być najpierw przekształcone do postaci alternatywy wykluczającej termów. Najprostszym podejściem mogłoby być wyliczenie całej tabeli funkcji i przedstawienie jej jako alternatywa wykluczająca mintermów.
\par Weźmy następująco zdefiniowaną funkcję
\[f(\vec{x}) = and(x_0, h(\vec{x}))\]
gdzie $\vec{x}$ jest wektorem długości $n \in \mathbb{N}$, a $h$ funkcją stała równą 1.\\
Do wyrażenia tej funkcji potrzeba $\frac{1}{2}n$ mintermów, które zostaną zamienione na $\frac{1}{2}n$ bramek Toffoliego. Jest to wysoko nieoptymalne, kiedy do obliczenia tej funkcji wystarczy jedna bramka Feymana.
\[
    \Qcircuit @C=1.5em @R=1.5em {
        & \lstick{\ket{x_0}} & \ctrl{4} & \rstick{\ket{a}} \qw & \\
        & \lstick{\ket{x_1}} & \qw & \rstick{\ket{a}} \qw & \\
        \vdots & \vdots & & \vdots & \vdots \\
        & \lstick{\ket{x_n}} & \qw & \rstick{\ket{a}} \qw & \\
        & \lstick{\ket{0}} & \targ & \rstick{\ket{f(\vec{x})}} \qw &
    }
\]
Zamiast wyliczać całą tabelę wartościowań dla funkcji boolowskiej można zbudować drzewo decyzyjne. 
\begin{definition}
    Rozwinięcie Shannona dla funkcji boolowskiej $f$ względem zmiennej $x_i$ denifiuje się następująco
    \[f(x_0, x_1, \ldots x_n) = x_i*f_{x_i} \otimes \overline{x_i}*f_{\overline{x_i}}\]
    gdzie 
    \[f_{x_i} = f(x_i = 1)\]
    \[f_{\overline{x_i}} = f(x_i = 0)\]
    To znaczy, $f_{x_i}$ jest funkcją powstałą przez podstawienie za $x_i$ $1$.
\end{definition}
\subsection{Algorytm}
\par Korzystając z rozwinięć Shannona budujemy drzewo w następujący sposób:
\begin{enumerate}
    \item Podstaw pod korzeń funkcję $f$. Niech $f' := f$.
    \item Jeśli $f'$ nie jest funkcją stałą przejdź do kolejnego kroku, w przeciwnym przypadku koniec.
    \item Niech $a$ będzie dowolnym z argumentów funkcji $f'$.
    \item Podstaw jako prawego syna węzła z $f'$ $f'_{a}$, jako lewego $f'_{\overline{a}}$.
    \item Wykonaj kroki od 2. dla synów $f'$.
\end{enumerate}
Algorytm ten zwraca drzewo kolejnych rozwinięć o następującej strukturze:
\begin{center}
    \Tree[.$f$ 
        [.$f_{x_0}$ 
            [.$f_{x_0, x_1}$ [.{\vdots} ] [.{\vdots} ]] 
            [.$f_{x_0, \overline{x_1}}$ [.{\vdots} ] [.{\vdots} ]]
        ] 
        [.$f_{\overline{x_0}}$ 
            [.$f_{\overline{x_0}, x_1}$ [.{\vdots} ] [.{\vdots} ]]
            [.$f_{\overline{x_0}, \overline{x_1}}$ [.{\vdots} ] [.{\vdots} ]]
        ]
    ]   
\end{center}
Korzystając z tak zdubowanego drzewa można łatwo znaleźć postać alternatywy wykluczającej termów funkcji $f$. Każdy liść zawiera funkcję stałą, której odpowiada term. Na przykład dla liścia $f_{x_0, \overline{x_1}, x_2}$ mamy term $x_0\overline{x_1}x_2$. Wartość tej funkcji jest wskaźnikiem czy dany term należy do funkcji $f$.
\subsubsection{Przykład} 
Dla przykładowej funkcji $f(a,b,c) = a \land (b \lor \neg c)$ algorytm stworzy następujące drzewo:
\begin{center}
    \Tree[.$f$ 
        [.{$f_{a} = b \lor \neg c$} 
            [.{$f_{a, b} = 1$} ] 
            [.{$f_{a, \overline{b}} = \neg c$}
                [.{$f_{a, \overline{b}, c} = 0$} ]
                [.{$f_{a, \overline{b}, \overline{c}} = 1$} ]
            ] 
        ] 
        [.{$f_{\overline{a}} = 0$} ]
    ]   
\end{center}
Stąd 
\[f(a, b, c) = ab \otimes a\overline{b}\overline{c}\]